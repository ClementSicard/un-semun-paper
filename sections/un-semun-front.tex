
\subsection{\texttt{un-semun-front}: A React \& Sigma.js frontend} \label{ssec:un-semun-front-a-react-sigma-js-frontend}

I used React combined with Typescript for the UI framework, as well as \href{https://chakra-ui.com/}{Chakra UI} for the UI components and \href{https://www.sigmajs.org/}{\texttt{Sigma.js}}, via its React adapter \href{https://sim51.github.io/react-sigma/}{\texttt{@react-sigma}} for the network map. \href{https://graphology.github.io/}{\texttt{graphology}} was also used for graph manipulation in the frontend, mostly to iterate over graph elements to perform styling. The code is available here: \repo{un-semun-front}.


\begin{figure}[!htb]
    \centering

    \includegraphics[width=\linewidth]{res/un-frontend.png}
    \caption{Screenshot of the frontend on the query \texttt{"Women in peacekeeping"}.}
    % change caption position to the bottom

    \label{fig:frontend-screenshot}
\end{figure}


The frontend was the part I was the least familiar with, but Chakra UI allowed to insert nice-looking components that I could customize based on my use case. It is composed in two panes:

\begin{itemize}
    \item \textbf{The search bar} (on top): the user can enter its prompt, which will be sent to the API (\ref{ssec:un-semun-api-an-api-for-un-semun-front-using-fastapi}) to retrieve the results.

    \item \textbf{The result list} (on the left). The results are displayed as a scrollable list of \texttt{Card} components, with the title, the summary, UNDL unique identifier, subjects, the date of publication and extracted UN bodies. The user easily go to the corresponding UNDL website page, or directly read the corresponding English PDF file.

    \item \textbf{The network map} (on the right). The results of the search are also displayed as a network map, with the documents, related United Nations bodies, topics from UNBIS Thesaurus taxonomy, and named entities extracted from the documents. The user can click on a node to display the document in the left pane. This map is also fetched using the API (\ref{ssec:un-semun-api-an-api-for-un-semun-front-using-fastapi}) and is based on the results of the machine learning pipeline (\ref{ssec:un-ml-pipeline-the-machine-learning-pipeline}).
\end{itemize}

